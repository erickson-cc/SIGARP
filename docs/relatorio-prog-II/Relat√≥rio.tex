\documentclass[ ]{article}
\usepackage{indentfirst}
\title{Relatório Trabalho Integrador}
\date{18 de Julho de 2025}
\author{Aysha Thayná, Erickson Müller}
\begin{document}
	\maketitle
	\newpage
	\section{Visão geral do projeto}
	\section{Objetivos do sistema}
		Manter um controle e histórico de licitações e 
	\section{Tecnologias utilizadas}
		\begin{itemize}
			\item \textbf{Backend:} Node.js com Express.
			\item \textbf{Frontend:} HTML, CSS e Javascript.
			\item \textbf{Banco de Dados:} PostgreSQL.
			\item \textbf{Autenticação:} Passport.js
		\end{itemize}
	\section{Arquitetura e lógica de implementação}
		A estrutura de repositórios dentro da pasta $/src/$ foi separada em $/back\_end/$, $/front\_end/$ e $/database/$. 
		\subsection{Arquitetura Cliente-Servidor}
			Em $/back\_end/$ estão os arquivos do servidor node iniciado com o yarn. A lógica do servidor está no arquivo \textit{server.js}, o servidor expõe endpoints RESTful API para manipulação dos dados com o banco de dados através do \textit{pg-promise}.
			
			No $/front\_end/$, páginas em HTML são renderizadas pelo navegador usando a lógica interativa do Javascript e sendo estilizadas por CSS. Então existem 3 subpastas: $/html/$, $/css/$ e $/js/$.
			
			Em $/database/$ está o arquivo de criação do banco de dados $dbSIGARP.sql$, assim como outros arquivos de inserção de tuplas no banco. Para acessar o sistema, deve-se rodar o script do arquivo $dbCREATE-Admin.sql$ que cria o usuário administrador com seu devido login e senha.
			
			
		\subsection{Criptografia e Autenticação}
		Para autenticação, o módulo Passport trabalha em harmonia com o JSON Web Token. O bcrypt faz a comparação da senha inserida com a armazenada no banco de dados e a função isAuthenticated autoriza a rota.

	\section{Detalhamento de lógica por funcionalidade}
		Para a autenticação, o arquivo login.html coleta as credenciais do usuário, a requisição de login é enviada ao backend e este verifica as credenciais na tabela usuario do banco de dados, com a senha criptografada.
		
		Para o cadastro de novas licitações, o arquivo cadastro\_licitacao.html apresenta um formulário para que o usuário insira os dados, em seguida realiza as validações básicas no lado do cliente.
	\section{Estrutura de banco de dados}
	\section{Facilidades e dificuldades encontradas}
		\section{Facilidades}
		\subsection{Dificuldades}
	\section{Repositório do Github}
	\section{Conclusão}
	
\end{document}