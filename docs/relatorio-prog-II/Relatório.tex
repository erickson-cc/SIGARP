\documentclass[ ]{article}
\usepackage{indentfirst}
\title{Relatório Trabalho Integrador\\Sistema de Gerenciamento\\de Atas de Registro de Preços}
\date{18 de Julho de 2025}
\author{Aysha Thayná\\ Erickson Müller}
\begin{document}
	\maketitle
	\newpage
	\section{Visão geral do projeto}
		O SIGARP - Sistema de Gerenciamento de Ata de Registros de Preços será utilizado pela Regional Oeste da Secretaria de Estado de Justiça e Reintegração Social para administrar o andamento de licitações e disponibilizar a consulta por diversos setores.
	\section{Objetivos do sistema}
		Manter um controle e histórico de licitações, possibilitando o cadastro e atualização de itens, fornecedores, lances, pedidos e demandas. As diversas licitações em processo silmutâneo podem ser verificadas no sistema.
		
		O objetivo deste sistema não é substituir a rotina administrativa vigente, mas sim auxiliar na comunicação das situações de atas em diversos setores que estão espalhados em mais de 6 cidades.
		
	\section{Tecnologias utilizadas}
		\begin{itemize}
			\item \textbf{Backend:} Node.js com Express; Passport.js, bcrypt; pg-promise.
			\item \textbf{Frontend:} HTML, CSS e Javascript; Axios, jQuery.
			\item \textbf{Banco de Dados:} PostgreSQL.
			%\item \textbf{Autenticação:} Passport.js, bcrypt.
		\end{itemize}
	\section{Arquitetura e lógica de implementação}
		A estrutura de repositórios dentro da pasta $/src/$ foi separada em $/back\_end/$, $/front\_end/$ e $/database/$. 
		\subsection{Arquitetura Cliente-Servidor}
			Em $/back\_end/$ estão os arquivos do servidor node iniciado com o yarn. A lógica do servidor está no arquivo \textit{server.js}, o servidor expõe endpoints RESTful API para manipulação dos dados com o banco de dados através do \textit{pg-promise}.
			
			No $/front\_end/$, páginas em HTML são renderizadas pelo navegador usando a lógica interativa do Javascript e sendo estilizadas por CSS. Então existem 3 subpastas: $/html/$, $/css/$ e $/js/$.
			
			Em $/database/$ está o arquivo de criação do banco de dados $dbSIGARP.sql$, assim como outros arquivos de inserção de tuplas no banco. Para acessar o sistema, deve-se rodar o script do arquivo $dbCREATE-Admin.sql$ que cria o usuário administrador com seu devido login e senha.
			
			
		\subsection{Criptografia e Autenticação}
		Para autenticação, o módulo Passport trabalha em harmonia com o JSON Web Token. O bcrypt faz a comparação da senha inserida com a armazenada no banco de dados e a função isAuthenticated() autoriza a rota.

	\section{Detalhamento de lógica por funcionalidade}
		Para a autenticação, o arquivo login.html coleta as credenciais do usuário, a requisição de login é enviada ao backend e este verifica as credenciais na tabela usuario do banco de dados, com a senha criptografada.
		
		Para o cadastro de novas licitações, itens, usuários, unidades, etc o arquivo de cadastro obedece ao padrão de nome cadastro\_xxxx.html e apresenta um formulário para que o usuário insira os dados.%, em seguida realiza as validações básicas no lado do cliente.
		
		O Javascript recebe essas informações pelo arquivo \texttt{script\_cadastro\_XXXX.js} através da biblioteca JQuery e chama uma função \texttt{createPost\_XXXX()} que, usando a bibliotaca Axios, manda os dados para o post no backend através do link \texttt{/cadastro\_XXXX\_rota}.
		
		O post fará a validação da requisição, em seguida insere a tupla no banco de acordo com os dados da rota. A situação de cada etapa do post pode ser analisada no terminal do servidor.
		
		Para consultar as licitações cadastradas, o arquivo \texttt{consulta\_licitacoes.html} exibe uma tabela e, ao carregar a página, um script \texttt{script\_consulta\_licitacoes.js} cria um array vazio faz uma requisição GET para o backend, via Axios. O backend consulta a tabela \texttt{licitacao} no banco de dados e retorna a lista de licitações em formato JSON. O Javascript do frontend, então, recebe esses dados e os renderiza na tabela da página usando a função renderizarLicitacoes().
		
		Cada linha dessa tabela possui um botão \textbf{editar} e \textbf{remover}.		 Dentro da função renderizarLicitacoes(), o click é reconhecido e enviado para as determinadas rotas. Ao clicar em \textbf{editar}, o sistema nos direciona para a página $/editar\_licitacao/$ com o número e ano da licitação a ser editada. Um processo similar ao de cadastro está presente nessa tela, o usuário poderá alterar a descrição da licitação e a alteração é efetuada pelo método PUT. Em seguida o usuário é redirecionado novamente para a tela de consulta de licitações.
		
		Ao clicar em remover, a função \textbf{removerLicitacao()} é invocada, um alert aparece na tela e o \textit{delete} é solicitado no backend com o ano e número da licitação. O método delete exclui a licitação do banco de dados.
	%\section{Estrutura de banco de dados}
	\section{Facilidades e dificuldades encontradas}
		\subsection{Facilidades}
			Neste projeto, usamos a stack ensinada em aula de Programação II, isso facilitou muito no início do projeto pois usamos o template do esqueleto de um sistema para fazer o CRUD e a autenticação/autorização.
		\subsection{Dificuldades}
			O modelo de banco de dados escolhido é grande e complexo, o que acaba exigindo um grande número de telas no sistema para fazer essa integração.
	\section{Repositório do Github}
	https://github.com/erickson-cc/SIGARP.git
	%\section{Conclusão}
	
\end{document}